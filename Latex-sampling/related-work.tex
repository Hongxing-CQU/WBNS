\section{Related work}

\subsection{Blue noise sampling}
Blue noise distribution was introduced to solve
problems of image anti-aliasing~\cite{dippe:1985:antialiasing}. For
the huge applied values of its natural property (random and uniform
distributions which matches the distribution of retina cell on human eyes),
it has been widely used to solve diverse kinds of problems in computer
graphics.
There are three classical methods to generate blue noise
distributions.
The first one is stochastic sampling.
Dart throwing was firstly introduced by
Cook in 1986~\cite{cook:1986:stochastic}. During the sampling, dart throwing positions
samples one by one randomly and accepts a new sample just when there are no
other samples within a disk of given radius surrounding it. The basic idea
is simple and easy to implement but extremely time-consuming, it even might
not terminate. Since then, many variations of dart throwing were proposed to
solve the problem~\cite{Mitchell:1987:generating,dunbar:2006:spatial,white:2007:poisson,wei:2008:parallel,gamito:2009:accurate,ebeida:2011:efficient}.
However, it is difficult to control the number of points generated with these method,
and produced distribution is also with relatively high variance
by current standards.

Optimal sampling is another type of methods to generate higher quality sampling.
A well-known approach in this class is Lloyd relaxation~\cite{lloyd:1982:least}.
The basic idea of this method
is to relax the original distribution to make it obtain blue noise
characteristics and improve the spectrum properties~\cite{mccool:1992:hierarchical}.
However, the produced samples suffer from too much regularity which causes aliasing problem.
% First, a original sampling distribution
%is obtained by traditional sampling methods. Second, the original ~\cite{balzer:2009:capacity} \cite{de:2012:blue}
%distribution is relaxed by Lloyd relaxation\cite{lloyd:1982:least} .
%Finally, the distribution with blue noise characteristics is generated. The
To improve the irregularity,
Baler et al.~\shortcite{balzer:2009:capacity} proposed a variant of Lloyd's method with capacity constraint(CCVT)
which enforces that each point obtains equal importance in the distribution.
~\cite{xu:2011:capacity} proposed Capacity-Constrained Delaunay Triangulation (CCDT) by replacing Voronoi cell of CCVT with Delaunay triangulation.
Furthermore, CCDT is extended to Capacity-constrained Surface Triangulation(CCST)~\cite{xu:2012:blue} for surface sampling.
De Goes et al.~\cite{de:2012:blue} considered CCVT as an optimal transport problem in
the space of power diagrams.
In addition to Lloyd's method,
the kernel density model is also applied to optimal sampling.
In~\cite{fattal:2011:blue},
Fattal used a radially-symmetric kernel function to produce an approximated density function.
The difference between the approximation and the given target point density function assigns an energy value to the points configuration.
Jiang et al.~\shortcite{jiang:2015:blue} used a kernel function to apply the smoothed particle hydrodynamics method for a variety of controllable blue noise sample patterns.
Compared with \cite{de:2012:blue},
our approach is generalized as the minimization of Wasserstein distance
by using the transport plan instead of the partition in \cite{de:2012:blue}.
Thus, our approach is feasible for multi-class sampling.


\subsection{Multi-class sampling}
Wang and Parker~\shortcite{Wang:1999:BlueNoise} studied the power spectrum characteristics of combined blue noise patterns and proposed a multi-class blue noise sampling approach on digital filter techniques.
However, the approach is constrained to uniform,regular and discrete sampling.
Wei\shortcite{wei:2010:multi}, Schmaltz et al.\shortcite{schmaltz:2012:multi} and Jiang et al.\shortcite{jiang:2015:blue} proposed a multi-class sampling
approach by enforcing blue noise property via an interaction matrix which
encodes the spacing between class pairs.
However, the matrix can exhibit discontinuous changes in the off-diagonal entries.
Chen et al.~\shortcite{chen:2012:variational} proposed a multi-class sampling strategy based on capacity-constrained Voronoi tessellation.
Two stage algorithm
 is applied to optimization of point distributions.
The first stage is the optimal procedure for each individual class,
and the second stage is the optimal procedure for the combined class.
In our approach,
the combined class is directly treated as a single class.
Normalized parameters are used to balance blue noise properties of different classes.
Compared with~\cite{Wang:1999:BlueNoise},
our approach preserves blue noise patterns in the density space other than in the spectrum space.
Compared with~\cite{chen:2012:variational},
our approach optimizes the point distribution of each individual class and that of the combined class simultaneously.

\begin{figure}[htb]
\centering
\subfigure[]{
\label{one-class-a}
   \includegraphics[width=.4\textwidth]{figure/one-class-1024.png}
}
\subfigure[]{
\label{one-class-b}
   \includegraphics[width=.45\textwidth]{figure/ramp-1.png}
}
\caption{
Comparisons of CCVT, BOT and our algorithm for a single class blue noise sampling.
(a) shows the comparison for the case of constant density.
(b) shows the comparison on for the case of a quadratic density function. }
\label{one-class-sampling}
\end{figure}


\begin{figure*}[htb]
\centering
\subfigure[{ Dart throwing [Wei 2010]}]{
   \includegraphics[width=0.45\textwidth]{figure/2-weiliyi.png}
   \label{two-class-sampling-a}
}%
\hspace{0.1in}
\subfigure[{ Our algorithm:$\lambda_{1,2,3}=(1,1,1)/3$}]{
   \includegraphics[width=.45\textwidth]{figure/2-111.png}
   \label{two-class-sampling-a}
}\\
\subfigure[{ Our algorithm:$\lambda_{1,2,3}=(1,1,2)/4$}]{
   \includegraphics[width=.45\textwidth]{figure/2-112.png}
   \label{two-class-sampling-c}
}%
\hspace{0.1in}
\subfigure[{ Our algorithm:$\lambda_{1,2,3}=(1,1,6)/8$}]{
   \includegraphics[width=.45\textwidth]{figure/2-116.png}
}%
\caption{The comparison of Wei's algorithm and our algorithm for two-class blue noise sampling.
In our algorithm, $\lambda_1$,$\lambda_2$,$\lambda_3$ are the weighted parameters for red samples, blue samples and the combined samples.
The number of samples for each class is 1000.
The weighted parameter $\lambda_i$ makes an influence on the trade-off between good distribution of the samples belonging to each individual class and good distribution of total samples.
Large $\lambda_i$ preserves the blue noise property of the $i$-$th$ well.
}\label{two-class-sampling}
\end{figure*}
\subsection{Wasserstein barycenter}
For completeness, we briefly review Wasserstein barycenter in the next.
By analogy with the barycenter of points $(x_1,...,x_n)$ in the Euclidean space, which is obtained as the minimizer of $x\leftarrow\sum_{i=1}^n||x-x_i||^2$,
Wasserstein barycenter is defined as the barycenter of probability measure $\nu_1,...,\nu_n$ in the Wasserstein space
by replacing the Euclidean distance with the Wasserstein distance.

For an arbitrary space $\Omega$,
we use $d:\Omega\times\Omega\rightarrow{\mathbb{R}}_+$ to denote the distance metric,
so $d(x,y)$ is the shortest distance form $x$ to $y$ along $\Omega$.
We use $P(\Omega)$ to indicate the set of probability measures on $\Omega$,
and $P(\Omega\times\Omega)$ to indicate the set of probability measures on the product space $\Omega\times\Omega$.
%\begin{equation*}
% P(\Omega)=\{\mu|\mu(\phi)=0,\mu(\Omega)=1, 0\leq\mu(U)\leq 1 (U\subseteq\Omega)\}.
%\end{equation*}
%For any point $x\in\Omega$, $\delta_x$ is the Dirac unit mass on $x$.

For probability measures $\mu$ and $\nu$ in $P(\Omega)$,
the $p$-Wasserstein distance~\cite{villani:2008:optimal} between $\mu$ and $\nu$ is defined as
\begin{equation*}
  W_p(\mu,\nu)=\left(\inf\limits_{\pi\in\prod(\mu,\nu)}\int_{\Omega\times\Omega}d(x,y)^pd\pi(x,y)\right )^{1/p}, (p\geq1)
\end{equation*}
where $\prod(\mu,\nu)$ is the subset in $P(\Omega\times\Omega)$, and meets mass conservation laws
\begin{equation*}
\prod(\mu,\nu)=\{\pi\in P(\Omega,\Omega)|\pi(\cdot,\Omega)=\mu,\pi(\Omega,\cdot)=\nu\}.
\end{equation*}
$\pi$ is a transportation plan,
which describes the amount of mass $\pi(x,y)$ to be placed from $\mu$ at $x$ towards $y$ to create $\nu$ in aggregation.
Thus,
Wasserstein distance describes the minimum cost of transporting the source $\mu$ to the target $\nu$.

% $\mu(U)=\int_{U\in\Omega}\rho(x)dx$, $(\rho(x)\geq0)$
% is a probability density function on the space $\Omega$ and $\int_{\Omega}\rho(x)dx=1$.
%$x,y\in\Omega$, $d(x,y)$ is the cost for transporting one unit of mass from $x$ to $y$, $\prod(\mu,\nu)$ is the set of all probability measures on $\Omega^2$
 % that have magrinals $\mu$ and $\nu$.

Based on Wasserstein distance,
Wasserstein barycenter~\cite{agueh:2011:barycenters,cuturi:2013:fast,BTSSPP:2016:Wasserstein} of $N$ probability measures ${\nu_1,...,\nu_N}$ in $P(\Omega)$ is defined as:
\begin{equation}\label{WB}
%\begin{split}
    \mu=\arg\min\limits_\mu\sum\limits_{i=1}^N\lambda_iW_p^p(\mu,\nu_i)  \quad
    s.t. \sum\limits_{i=1}^N\lambda_i=1, \lambda_i\geq0
%\end{split}
\end{equation}
Wasserstein barycenter is an optimal probability measure $\mu$ to approximate probability measures ${\nu_1,...,\nu_N}$ with weights in Wasserstein space.
If $\mu$ is a discrete probability measure,
$\mu$ represents sampling points which approximate probability measures ${\nu_1,...,\nu_N}$.
%is sampling with capacity constrain in the space $\Omega$.
Thus,
Wasserstein barycenter is a general framework for sampling in the space $\Omega$.








