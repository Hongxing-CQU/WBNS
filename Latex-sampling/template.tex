%%% template.tex
%%%
%%% This LaTeX source document can be used as the basis for your technical
%%% paper or abstract. Regardless of the length of your document, the commands
%%% are all the same.
%%%
%%% The "\documentclass" command is the first command in your file. If you want to
%%% prepare a version of your article with line numbers - a "review" version -
%%% include the "review" parameter:
 \documentclass[review]{acmsiggraph}
%%%
%\documentclass{acmsiggraph}
\usepackage{subfigure}
\usepackage{graphicx}
\usepackage{multirow}
\usepackage{amsfonts}

%%% Title of your article or abstract.
\title{Wasserstein Blue Noise Sampling }
%\title{Adaptive multi-class blue noise sampling on Wasserstein barycenter}
\author{papers_0229}
%\author{Hongxing Qin\thanks{e-mail:qinhx@cqupt.edu.cn},
 %Yunhai Wang\thanks{e-mail:cloudseawang@gmail.com},
 %Baoquan Chen\thanks{e-mail:baoquan@sdu.edu.cn} \\
% Chongqing Key Laboratory of Computational Intelligence,
%             Chongqing University of Posts \& Telecommunications, Chongqing, China\\
% Shandong University\\
%}
%\author{Yunhai Wang\thanks{e-mail:cloudseawang@gmail.com}\\Shandong University}
%\author{Baoquan Chen\thanks{e-mail:baoquan@sdu.edu.cn}\\Shandong University}
\pdfauthor{Stephen N. Spencer}

%%% Used by the ``review'' variation; the online ID will be printed on
%%% every page of the content.

\TOGonlineid{229}

% User-generated keywords.

\keywords{blue noise sampling,  Wasserstein barycenter, adaptive sampling, surface sampling}

% With the "\setcopyright" command the appropriate rights management text will be added
% to your document.

%\setcopyright{none}
%\setcopyright{acmcopyright}
%\setcopyright{acmlicensed}
\setcopyright{rightsretained}
%\setcopyright{usgov}
%\setcopyright{usgovmixed}
%\setcopyright{cagov}
%\setcopyright{cagovmixed}
%\setcopyright{rightsretained}

%The year of publication in the "\copyrightyear" command.

\copyrightyear{2016}

%%% Conference information, from the completed rights management form.
%%% The "\conferenceinfo" command has two parameters:
%%%    - conference name
%%%    - conference date and location
%%% The "\isbn" field includes the year and month after the article ISBN.

\conferenceinfo{SIGGRAPH 2016 Posters}{July 24-28, 2016, Anaheim, CA}
\isbn{978-1-4503-ABCD-E/16/07}
\doi{http://doi.acm.org/10.1145/9999997.9999999}

\begin{document}

%%% This is the ``teaser'' command, which puts an figure, centered, below
%%% the title and author information, and above the body of the content.
\teaser{
\begin{center}
  \centering
\begin{minipage}{1\textwidth}
  \includegraphics[width=0.135\textwidth]{figure/adaptive-size-123.png}
  %\caption{abc}
  %\end{minipage}
  %  \begin{minipage}{0.25\textwidth}
  \includegraphics[width=0.135\textwidth]{figure/adaptive-size-12.png}
  %\end{minipage}
   % \begin{minipage}{0.23\textwidth}
  \includegraphics[width=0.135\textwidth]{figure/adaptive-size-13.png}
  %\end{minipage}
  %  \begin{minipage}{0.23\textwidth}
  \includegraphics[width=0.135\textwidth]{figure/adaptive-size-23.png}
    \includegraphics[width=0.135\textwidth]{figure/adaptive-size-1.png}
  %\end{minipage}
   % \begin{minipage}{0.23\textwidth}
  \includegraphics[width=0.135\textwidth]{figure/adaptive-size-2.png}
  %\end{minipage}
  %  \begin{minipage}{0.23\textwidth}
  \includegraphics[width=0.135\textwidth]{figure/adaptive-size-3.png}
  \end{minipage}\\
  \centering
 \begin{minipage}{1\textwidth}
  \includegraphics[width=0.24\textwidth]{figure/bumpy-sphere123.png}
  %\end{minipage}
  %  \begin{minipage}{0.25\textwidth}
  \includegraphics[width=0.24\textwidth]{figure/bumpy-sphere1.png}
  %\end{minipage}
   % \begin{minipage}{0.23\textwidth}
  \includegraphics[width=0.24\textwidth]{figure/bumpy-sphere2.png}
  %\end{minipage}
  %  \begin{minipage}{0.23\textwidth}
  \includegraphics[width=0.24\textwidth]{figure/bumpy-sphere3.png}
  \end{minipage}
  \caption{Multi-class blue noise sampling  with different sizes on three same constant density functions.
  Sample distribution with larger size remains blue noise pattern,
  and sample distribution with smaller size might deviate from blue noise pattern.
  The number of samples for each class is $62$, $250$, $1000$ (top) and $100$, $300$, $900$(bottom).
  The weighted parameters are set as $\lambda_{1,2,3,4}=(1,1,1,1)/4$.
  % with$62, 250, 1000$ $100$,
  % $300$ and $900$ points for each class.
  }\label{adaptive-size-sampling}
\end{center}
}
\maketitle

\begin{abstract}
Sampling is a core component for many applications such as imaging,
rendering, geometry processing and visualization. Prior research
has mainly focused on blue noise sampling with a single
class of samples. Limited research has been done on multi-class
sampling; especially, multi-class blue noise sampling is still a challenging problem
when the density distribution of every class is nonuniform and different from each
other. In this paper, we present a Wasserstein blue
noise sampling algorithm for multi-class sampling by throwing samples
as Wasserstein barycenter of multiple density distributions. We
employ a more general representation of optimal transport problem
to break the partition necessary in other optimal sampling.
Moreover, adaptive blue noise distribution property for every
individual class is guaranteed, as well as their combined class.
Furthermore, the efficiency of sampling is improved by applying
Wasserstein distance with entropic constraints. Our method can
be applied to multi-class sampling on the point set surface. We also
demonstrate applications in object distribution and color stippling.

%Sampling is a core component for many applications such as imaging,
%rendering, geometry processing and visualization.
%Prior research has mainly focused on blue noise sampling with a single class of samples.
%Even if a few work has been done on multi-class sampling,
%multi-class blue noise sampling is a challenge problem when the density distribution of every class is nonuniform and different
%with others.
%In the paper,
%we present a Wasserstein blue noise sampling algorithm for multi-class samples by throwing samples as Wasserstein barycenter of multiple density distribution.
%We applied a more general representation of optimal transport problem to break the partition which is necessary in other optimal sampling.
%On the one hand,
%adaptive blue noise distribution property for every individual class is guaranteed as well as their combined class.
%On the other hand, the efficiency of sampling is improved by applying Wasserstein distance with entropic constraints.
%Our method can also be applied for multi-class sampling on the point set surface.
%We demonstrate applications in object distribution and  color stippling.

\end{abstract}



%\begin{abstract}
%Sampling is a core component for many applications such as imaging,
%rendering, geometry processing and visualization.
%Prior research has mainly focused on blue noise sampling with a single class of samples.
%Even if a few work has been done on multi-class sampling,
%multi-class blue noise sampling is a challenge problem when the density distribution of every class is nonuniform and different
%with others.
%In the paper,
%we present an Wasserstein blue noise sampling algorithm for multi-class samples by throwing samples as Wasserstein barycenter of multiple density distribution.
%On the one hand,
%adaptive blue noise distribution property for every individual class
%is guaranteed as well as their combined class.
%On the other hand, the efficiency of sampling is improved by applying Wasserstein distance with entropic constraints.
%At the same time, our method can also be applied for multi-class sampling with changed size and multi-class sampling on the point set surface.
%We demonstrate applications in object distribution and  color stippling.

%\end{abstract}
%
% The code below should be generated by the tool at
% http://dl.acm.org/ccs.cfm
% Please copy and paste the code instead of the example below.
%
\begin{CCSXML}
<ccs2012>
<concept>
<concept_id>10010147.10010371.10010382</concept_id>
<concept_desc>Computing methodologies~Image manipulation</concept_desc>
<concept_significance>500</concept_significance>
</concept>
<concept>
<concept_id>10010147.10010371.10010382.10010236</concept_id>
<concept_desc>Computing methodologies~Computational photography</concept_desc>
<concept_significance>300</concept_significance>
</concept>
</ccs2012>
\end{CCSXML}

\ccsdesc[500]{Computing methodologies~Image manipulation}
\ccsdesc[300]{Computing methodologies~Computational photography}

%
% End generated code
%

% The next three commands are required, and insert the user-generated keywords,
% The CCS concepts list, and the rights management text.
% Please make sure there is a blank line between each of these three commands.

\keywordlist

\conceptlist

\printcopyright

\section{Introduction}
Sampling is a ubiquitous problem for many graphics application~\cite{Lagae:2008:CPDD},
such as rendering~\cite{Mitchell:1987:generating},
imaging~\cite{Robert:1988:dithering},
geometry processing~\cite{Oztireli:10:Spectral}
and visualization~\cite{chen:2014:visual}.
Today, there is
a large number of sampling patterns.
The blue noise sampling distribution yields superior
image quality over alternative distributions with the same
number of samples since the retina cell distribution of
a human being is a blue noise distribution~\cite{Yellott:1983:retina}.
Due to its importance,
blue noise sampling has been researched extensively in last decades.
Prior research mainly focus on a single class sampling.
In fact, multi-class of samples is involved in a variety of sampling problems ,
such as color stippling, visual abstraction of multi-class scatterplots,
and so on.
In these situations,
blue noise distribution property needs to be preserved for every individual class of samples
and their union simultaneously.
Although optimal sampling~\cite{balzer:2009:capacity,chen:2012:variational,de:2012:blue} generates good distribution with blue noise property,
it is difficult to directly extend them for multi-class blue noise sampling since
the partition is necessary in these method.
Failure case of multi-class sampling cannot be avoided~\cite{wei:2010:multi,chen:2012:variational}.


%However, the single class blue noise sampling method cannot directly be
%applied to these situation
%~\cite{wei:2010:multi},
%since the cut-off frequency of multi-class of samples
%is larger than that of each single classes of samples~\cite{Wang:1999:BlueNoise}.

Multi-class blue noise sampling is still a challenging problem,
although there are a few works on the issue recently.
Based on the analysis of power spectrum characteristics of combined classes,
digital filter techniques are applied to generate multi class samples in ~\cite{Wang:1999:BlueNoise}.
In their methods, $2^n-1$ kinds of combined patters need to be created,
where $n$ is the number of classes.
Thus, the problem tends to be more complicated with the increasing number of classes.
Dart throwing is extended to generate multi-class samples in ~\cite{wei:2010:multi}.
The main idea is to create a conflict matrix on accumulated two class sampling for explicit control of sample spacing.
However, it is a difficult to set the off-diagonal elements of the conflict matrix
to balance blue noise property of the individual class and that of the combined class
when the density distribution of every class is nonuniform and different with others.
In~\cite{chen:2012:variational},
Chen et al. extended blue noise sampling with capacity-constrained
Voronoi tessellation for multi-class sampling.
However,
The conflict between the Voronoi diagram of the combined class and the Voronoi diagram of each individual class
prohibits  optimizing the point distribution of
the combined class as well as the individual classes simultaneously.


In this paper, we present an Wasserstein blue noise sampling algorithm by
throwing samples as Wasserstein barycenter of multiple density distribution.
Blue noise property of every individual class and combined classes
is preserved by setting separately every individual density distribution and every combined density distribution as an independent density distribution to
compute the Wasserstein barycenter.
Our algorithm avoids the failure case of multi-class sampling on traditional Lloyd relaxation~\cite{wei:2010:multi}
 and on Voronoi tessellation~\cite{chen:2012:variational}  since
 the partition is broken in our method.
On the other hand,
the efficiency of sampling is improved by applying Wasserstein distance with entropic constraints.
At the same time,
our method can be directly applied for multi-class sampling on point set surface.

\subsection{Motivation and Contributions}
Although a series of algorithms on CCVT lead to high-quality blue noise samples,
it is very difficult to directly extend these algorithms to multi-class blue noise sampling.
The main reason is that the equi-capacity partitioning is key  to
ensure that each point convey equal visual importance in these algorithms.
It is questionable whether an adaptable partitioning for multi-class sampling can be made.
Thus, we must study a more general algorithm which is feasible for multi-class sampling,
adaptive sampling, and surface sampling.

In this paper, we present an adaptive multi-class blue noise sampling algorithm by
throwing samples as Wasserstein barycenter of multiple density distribution.
It computes point distributions to minimize Wasserstein distance.
%Blue noise property of every individual class and combined classes
%is preserved simultaneously.
Our experiments show that our result can generate high-quality blue noise spectrum for every individual class and the combined class.
Our contributions are as follows:
\begin{itemize}
\item We formulate multi-class sampling as a constrained Wasserstein barycenter of multiple density functions.
Constrained transport plans  are used to represent the capacity constrains exactly as the equi-capacity partitioning.
\item An iterative optimization procedure,
which combines stochastic gradient descent algorithm and Sinkhorn-Knopp��s matrix scaling algorithm,
not only break local spatial regularity, but also provide a fast and reliable numerical treatment.
\item The proposed approach can be easily extended to support adaptive sampling and general domain sampling,
such as sampling on point set surface.
\end{itemize}


\section{Related work}

\subsection{Blue noise sampling}
Blue noise distribution was introduced to solve
problems of image anti-aliasing~\cite{dippe:1985:antialiasing}. For
the huge applied values of its natural property (random and uniform
distributions which matches the distribution of retina cell on human eyes),
it has been widely used to solve diverse kinds of problems in computer
graphics.
There are three classical methods to generate blue noise
distributions.
The first one is stochastic sampling.
Dart throwing was firstly introduced by
Cook in 1986~\cite{cook:1986:stochastic}. During the sampling, dart throwing positions
samples one by one randomly and accepts a new sample just when there are no
other samples within a disk of given radius surrounding it. The basic idea
is simple and easy to implement but extremely time-consuming, it even might
not terminate. Since then, many variations of dart throwing were proposed to
solve the problem~\cite{Mitchell:1987:generating,dunbar:2006:spatial,white:2007:poisson,wei:2008:parallel,gamito:2009:accurate,ebeida:2011:efficient}.
However, it is difficult to control the number of points generated with these method,
and produced distribution is also with relatively high variance
by current standards.

Optimal sampling is another type of methods to generate higher quality sampling.
A well-known approach in this class is Lloyd relaxation~\cite{lloyd:1982:least}.
The basic idea of this method
is to relax the original distribution to make it obtain blue noise
characteristics and improve the spectrum properties~\cite{mccool:1992:hierarchical}.
However, the produced samples suffer from too much regularity which causes aliasing problem.
% First, a original sampling distribution
%is obtained by traditional sampling methods. Second, the original ~\cite{balzer:2009:capacity} \cite{de:2012:blue}
%distribution is relaxed by Lloyd relaxation\cite{lloyd:1982:least} .
%Finally, the distribution with blue noise characteristics is generated. The
To improve the irregularity,
Baler et al.~\shortcite{balzer:2009:capacity} proposed a variant of Lloyd's method with capacity constraint(CCVT)
which enforces that each point obtains equal importance in the distribution.
~\cite{xu:2011:capacity} proposed Capacity-Constrained Delaunay Triangulation (CCDT) by replacing Voronoi cell of CCVT with Delaunay triangulation.
Furthermore, CCDT is extended to Capacity-constrained Surface Triangulation(CCST)~\cite{xu:2012:blue} for surface sampling.
De Goes et al.~\shortcite{de:2012:blue} considered CCVT as an optimal transport problem in
the space of power diagrams (CCVT-PD).
In addition to Lloyd's method,
the kernel density model is also applied to optimal sampling.
In~\cite{fattal:2011:blue},
Fattal used a radially-symmetric kernel function to produce an approximated density function.
The difference between the approximation and the given target point density function assigns an energy value to the points configuration.
Jiang et al.~\shortcite{jiang:2015:blue} used a kernel function to apply the smoothed particle hydrodynamics method  (SPH) for a variety of controllable blue noise sample patterns.
Compared with \cite{de:2012:blue},
our approach is generalized as the minimization of Wasserstein distance
by using the transport plan instead of the partition in \cite{de:2012:blue}.
Thus, our approach is feasible for multi-class sampling.


\subsection{Multi-class sampling}
Wang and Parker~\shortcite{Wang:1999:BlueNoise} studied the power spectrum characteristics of combined blue noise patterns and proposed a multi-class blue noise sampling approach on digital filter techniques.
However, the approach is constrained to uniform,regular and discrete sampling.
Wei\shortcite{wei:2010:multi}, Schmaltz et al.\shortcite{schmaltz:2012:multi} and Jiang et al.\shortcite{jiang:2015:blue} proposed multi-class sampling
approaches by enforcing blue noise property via an interaction matrix which
encodes the spacing between class pairs.
However, the matrix can exhibit discontinuous changes in the off-diagonal entries.
Chen et al.~\shortcite{chen:2012:variational} proposed a multi-class sampling strategy based on capacity-constrained Voronoi tessellation.
Two stage algorithm
 is applied to optimization of point distributions.
The first stage is the optimal procedure for each individual class,
and the second stage is the optimal procedure for the combined class.
In our approach,
the combined class is directly treated as a single class.
Normalized parameters are used to balance blue noise properties of different classes.
Compared with~\cite{Wang:1999:BlueNoise},
our approach preserves blue noise patterns in the density space other than in the spectrum space.
Compared with~\cite{chen:2012:variational},
our approach optimizes the point distribution of each individual class and that of the combined class simultaneously.


\subsection{Wasserstein barycenter}
For completeness, we briefly review Wasserstein barycenter in the next.
By analogy with the barycenter of points $(x_1,...,x_n)$ in the Euclidean space, which is obtained as the minimizer of $x\leftarrow\sum_{i=1}^n||x-x_i||^2$,
Wasserstein barycenter is defined as the barycenter of probability measure $\nu_1,...,\nu_n$ in the Wasserstein space
by replacing the Euclidean distance with the Wasserstein distance.

For an arbitrary space $\Omega$,
we use $d:\Omega\times\Omega\rightarrow{\mathbb{R}}_+$ to denote the distance metric,
so $d(x,y)$ is the shortest distance form $x$ to $y$ along $\Omega$.
We use $P(\Omega)$ to indicate the set of probability measures on $\Omega$,
and $P(\Omega\times\Omega)$ to indicate the set of probability measures on the product space $\Omega\times\Omega$.
%\begin{equation*}
% P(\Omega)=\{\mu|\mu(\phi)=0,\mu(\Omega)=1, 0\leq\mu(U)\leq 1 (U\subseteq\Omega)\}.
%\end{equation*}
%For any point $x\in\Omega$, $\delta_x$ is the Dirac unit mass on $x$.

For probability measures $\mu$ and $\nu$ in $P(\Omega)$,
the $p$-Wasserstein distance~\cite{villani:2008:optimal} between $\mu$ and $\nu$ is defined as
\begin{equation*}
  W_p(\mu,\nu)=\left(\inf\limits_{\pi\in\prod(\mu,\nu)}\int_{\Omega\times\Omega}d(x,y)^pd\pi(x,y)\right )^{1/p}, (p\geq1)
\end{equation*}
where $\prod(\mu,\nu)$ is the subset in $P(\Omega\times\Omega)$, and meets mass conservation laws
\begin{equation*}
\prod(\mu,\nu)=\{\pi\in P(\Omega,\Omega)|\pi(\cdot,\Omega)=\mu,\pi(\Omega,\cdot)=\nu\}.
\end{equation*}
$\pi$ is a transportation plan,
which describes the amount of mass $\pi(x,y)$ to be placed from $\mu$ at $x$ towards $y$ to create $\nu$ in aggregation.
Thus,
Wasserstein distance describes the minimum cost of transporting the source $\mu$ to the target $\nu$.

% $\mu(U)=\int_{U\in\Omega}\rho(x)dx$, $(\rho(x)\geq0)$
% is a probability density function on the space $\Omega$ and $\int_{\Omega}\rho(x)dx=1$.
%$x,y\in\Omega$, $d(x,y)$ is the cost for transporting one unit of mass from $x$ to $y$, $\prod(\mu,\nu)$ is the set of all probability measures on $\Omega^2$
 % that have magrinals $\mu$ and $\nu$.

Based on Wasserstein distance,
Wasserstein barycenter~\cite{agueh:2011:barycenters,cuturi:2013:fast,BTSSPP:2016:Wasserstein} of $N$ probability measures ${\nu_1,...,\nu_N}$ in $P(\Omega)$ is defined as:
\begin{equation}\label{WB}
%\begin{split}
    \mu=\arg\min\limits_\mu\sum\limits_{i=1}^N\lambda_iW_p^p(\mu,\nu_i)  \quad
    s.t. \sum\limits_{i=1}^N\lambda_i=1, \lambda_i\geq0
%\end{split}
\end{equation}
Wasserstein barycenter is an optimal probability measure $\mu$ to approximate probability measures ${\nu_1,...,\nu_N}$ with weights in Wasserstein space.
If $\mu$ is a discrete probability measure,
$\mu$ represents sampling points which approximate probability measures ${\nu_1,...,\nu_N}$.
%is sampling with capacity constrain in the space $\Omega$.
Thus,
Wasserstein barycenter is a general framework for sampling in the space $\Omega$.

\section{Algorithm}

In the paper,
we take single class blue noise samples  as the Wasserstein barycenter of a probability measure.
Multi-class blue noise samples are chosen as the combined Wasserstein barycenter of multiple probability measures.
\subsection{Single class blue noise sampling}
Given an arbitrary space $\Omega$ and a density function $\varrho(x)$ on the space $\Omega$,
a probability measure is defined as:
\begin{equation}\label{probability-measure}
  \begin{split}
  \nu(U)=\int_U\varrho(x)dx, U\subseteq\Omega \\
  s.t. \int_{\Omega}\varrho(x)dx=1,\varrho(x)\geq0
  \end{split}
\end{equation}
Sampling the density function $\varrho(x)$ consists of picking a few representative points $x_i$
that capture $\varrho$ well.
These representative points $x_i$ represent a discrete probability measure $\mu$,
\begin{equation*}
\begin{split}
  \mu=\sum\limits_{i=1}^n\rho_i\delta_{x_i} \quad
  s.t. \sum\limits_{i=1}^n\rho_i=1,  \delta_{x_i}= \left\{ \begin{array}{ll}
  1 & x_i\in\Omega\\
  0 & \textrm{others}
  \end{array} \right.
  \end{split}
\end{equation*}
where $n$ is the number of samples and
$\rho_i$ is the probability at $x_i$.
Sampling, therefore, means that $\mu$ captures $\nu$ well.
We can use the Wasserstein barycenter to model a single class sampling as:
\begin{equation}\label{single-class-problem}
\begin{split}
 \mu=arg\min\limits_\mu W_p^p(\mu,\nu)=\inf\limits_\mu\int_{X\times\Omega}d(x_i,y)^pd\pi(x_i,y)\\
 s.t. \int_\Omega\pi(x_i,x)=\rho_i, \sum\limits_{i=1}^n\int_{U\subseteq\Omega}\pi(x_i,y)dy=\nu(U)
 \end{split}
\end{equation}
where $X=\{x_1,x_2,...,x_n\}\subseteq\Omega$.

Compared with previous methods on CCVT~\cite{balzer:2009:capacity,de:2012:blue},
the Wasserstein barycenter provides a more general framework for sampling.
Previous methods on CCVT are special cases of Eq.~\ref{single-class-problem} with an implicit constraint on  transport plan $\pi$,
that is
\begin{equation}\label{eq:transform-plan}
  \pi(x_i,y)\cdot\pi(x_j,y)=0 \quad(i\neq j).
\end{equation}
Eq.~\ref{eq:transform-plan} means that the mass in a certain domain can only be transported to one sampling point.
For example,
the corresponding domain of sampling points consists of a few discrete points in~\cite{balzer:2009:capacity},
and the power diagram region of sampling points in~\cite{de:2012:blue}.
In fact,
Eq.~\ref{eq:transform-plan} is a very strict condition for the variational problem in Eq.~\ref{single-class-problem}.
It can only be satisfied when $\nu$ is a continuous probability measure.
It also introduces a "failure case" for multi-class sampling~\cite{wei:2010:multi}.
We will discuss the reason in the next subsection.

\subsection{Multi-class blue noise sampling}
%For multi-class sampling,
%we will compute $N$-class sampling points for $m$-class probability density functions,
%where $N$ is the number of individual probability density functions and
%$m-N(m>N)$ is the number of combined probability density functions.
For multi-class blue noise sampling,
both each individual class and combined classes exhibit blue noise characteristics~\cite{wei:2010:multi}.
It is necessary to consider individual probability distributions and combined probability distributions simultaneously
in multi-class sampling. \\
%sampling points are drawn from $N$-class individual probability distribution
%and $M$-class combined probability distribution.
Let $\{\varrho_1,\cdots,\varrho_N\}$ be $N$ individual probability density functions
to be sampled on the space $\Omega$.
Let $\{\varrho_{{i}_1,...,i_k}\}(k\leq N, i\in\{N+1,N+2...,2^N-1\},i_j\in\{1,2,..,N\})$
be combined probability density functions of the combined classes,
which is combined by  the $i_1th,...,i_kth$ classes,
%A combined probability density function $\rho_{i_1,i_2,...,i_k}$,
% which describes distribution of m
% can be generated by $k(k\leq m)$ different probability density functions $\rho_{i_1},\rho_{i_2},...,\rho_{i_k}$,
% that is
\begin{equation*}
    \varrho_{i_1,i_2,...,i_k}=(\varrho_{i_1}+\varrho_{i_2}+...,+\varrho_{i_k})/k
\end{equation*}
Let $X_i$ be sampling points of the probability density function $\varrho_i$,
\begin{equation*}
  X_i=\{x_i^1,x_i^2,...,x_i^{n_i}\}\quad(1\leq i\leq N), x_i^j\in\Omega(1\leq j\leq n_i),
\end{equation*}
where $n_i$ is the number of sampling points $X_i$.

In terms of Eq.~\ref{probability-measure},
there is a corresponding probability measure $\nu_i(1\leq i\leq 2^N-1)$ for every $\varrho_i$ or  $\varrho_{{i_1,i_2,...,i_k}}$.
The probability measure $\nu_i(1\leq i\leq N)$ corresponds to $\varrho_i$,
and the probability measure $\nu_i(N<i\leq 2^N-1)$ corresponds to $\varrho_{i_1,...,i_k}$.
For sampling points $X_i$ or $\{X_{i_1}, X_{i_2},...,X_{i_k}\}$,
there is a corresponding discrete probability measure $\mu_i(1\leq i\leq 2^N-1)$.
For multi-class blue noise sampling,
every class sampling point $X_i$ captures the probability function $\varrho_i$ well,
and combined sampling points $\{X_{i_1}, X_{i_2},...,X_{i_k}\}$ also capture the probability density function $\varrho_{i_1,i_2,...,i_k}$ well.
Thus,
every discrete probability measure $\mu_i$ should approximate corresponding probability measure $\nu_i$ well.

Compared with a single class sampling,
we cannot compute an optimal $\mu_i$ for every $\nu_i$ in terms of Eq.~\ref{single-class-problem} independently,
since sampling points $X_i$ are influenced by individual probability measures $\nu_i$ and combined probability measuress $\nu_{i'}(i'_j=i)$ simultaneously.
We describe multi-class sampling as a combined Wasserstein barycenter,
\begin{equation}\label{eq:combined-wasserstein-barycenter}
 \bar\mu=arg\min\limits_{\bar\mu}\sum\limits_{i=1}^K\lambda_iW_p^p(\mu_i,\nu_i) \quad
  s.t.\sum\limits_{i=1}^K\lambda_i=1,(\lambda_i\geq0)
\end{equation}
where $\bar\mu=\{\mu_1,...,\mu_N,\mu_{N+1},...\mu_{K}\}$,
$\mu_i(1\leq i\leq N)$  is a discrete probability measure corresponding to $X_i$,
$\mu_i(N < i\leq K)$ is a discrete probability measure corresponding to combined sampling points and
$K-N$ is the number of combined classes.

In an extended version of~\cite{wei:2010:multi},
Wei pointed out that multi-class sampling on a traditional relaxation method~\cite{balzer:2009:capacity}
 might result in insufficient sample
uniformity for the reason below: for a given sample $s$,
different class combinations may have different opinions about the desired centroid
location to which $s$ should move.
In essence,
conflicts of desired centroid locations are introduced by the hard constraint on transport plans $\pi$ (Eq.~\ref{eq:transform-plan}).
In the traditional relaxation method~\cite{balzer:2009:capacity},
the mass in a certain domain can only be transported to the nearest sampling point.
Thus, the transport plan is local,
and the minimum unit of transported mass is constrained by the resolution of the domain.
The two limits make relaxing conflicts of desired centroid locations more difficult.

In order to preserve sufficient sample uniformity for multi-class sampling,
the constraint on the transport plan (Eq.~\ref{eq:transform-plan})
is broken through a more general transport plan $\pi$ in combined with a Wasserstein barycenter (Eq.~\ref{eq:combined-wasserstein-barycenter}).
On one hand,
$\pi$ is a separable transport plan in Eq.~\ref{eq:combined-wasserstein-barycenter},
which means the mass, located in a small domain, can be transported to different samples other than only one sampling point.
It extends the minimum unit of transported mass to computational accuracy.
Furthermore,
we relax transport plans via entropic regular terms as~\cite{cuturi:2013:sinkhorn}.
Regularized Wasserstein distance is defined as
\begin{equation}\label{eq:regulate-wasserstein-distance}
 W_p(\mu,\nu)=\left(\inf\limits_{\pi\in\prod(\mu,\nu)}\int_{\Omega^2}d(x,y)^pd\pi(x,y)+\epsilon H(\pi)\right )^{1/p}
\end{equation}
where $\epsilon$ is a positive regularization parameter,
and $H(\pi)$ is the entropy of $\pi$.
Entropic regularization terms expand transport range to the whole domain rather than the nearest transport in traditional relaxation method~\cite{balzer:2009:capacity}.
It provides more flexibility for reducing conflicts of the desired centroid.
Compared with multi-class sampling~\cite{wei:2010:multi},
our approach is an efficient relaxation method for multi-class blue noise sampling.

% other than a local transport plan.
%It make it possible to reduce the conflict in a more wide domain.
%In combined Wasserstein barycenter Eq.~\ref{eq:combined-wasserstein-barycenter},
%the constrain is broken via a general transport plan $\pi$.
%First,
%$\pi$ is a global transport plan other than a local transport plan,
%which is constrained in a local partition,
%such as discrete points~\cite{balzer:2009:capacity},
%Voronoi diagram~\cite{chen:2012:variational} or Power diagram~\cite{de:2012:blue}.
%Second,
%$\pi$ is a separable transport plan,
%which means the mass located in a small region can be transported to different samples other than only one sampling point.
%Furthermore,
%we relax transport plans via entropic regular terms as~\cite{cuturi:2013:sinkhorn}.

%More details can be found in~\cite{cuturi:2013:sinkhorn}.

%\subsection{Numerical Optimization}



\begin{figure}[htb]
\rule{0.5\textwidth}{0.4pt}
 \quad 01: //Wasserstein blue noise sampling\\
\quad 02: Input: domain $\Omega$, densities $\varrho_1,...,\varrho_N$, coordinates of discrete points $\mathbf{Y}$,
  number of sampling points $n_1,...,n_N$,
  number of combined classes $K-N$ and combined classes $\{i_1,...,i_k\}$\\
\quad 03: Output: coordinates $\mathbf{X}=\{x_1^1 \cdots  x_1^{N_1} \cdots x_n^1 \cdots x_n^{N_n}\}$  of sampled points\\
\quad 04: Initialize $\mathbf{X}$ with random points inside $\Omega$ conforming to $\rho_1,...,\rho_n$ \\
\quad 05: Initialize parameters $\epsilon$ for the entropic regularization\\
\quad 06: Initialize weight parameters $\lambda_i(1\leq i \leq K)$\\
%\quad 6: Initialize the parameter $t$ for the simulated annealing strategy\\
\quad 07: Initialize transport cost matrixes $\Pi_k=\frac{1}{nm}\mathbf{1}$, $\mathbf{1}$ is an all-1 matrix\\
\quad 08: Compute distance matrix $\mathbf D$\\
\quad 09: Compute the energy $E_0=\sum_{i=1}^K\lambda_i<\mathbf D,\mathbf{\Pi}_i>$\\
\quad 10: \textbf{Repeat}\\
11:\quad // Iterative bregman projections for $\mathbf{\Pi}_k$\\
12:\quad For i=1:K \\
13:\quad \quad  Iterative-bregman-projection() (lines 24-32) \\
14:\quad // Newton iterative method for $\mathbf{X}$ \\
15:\quad Generate K random numbers $r_1,...,r_K\in[0,1]$\\
16:\quad Normalize $r_1,...,r_K$,$\sum_{i=1}^Kr_i=1$\\
17:\quad Update $\mathbf X$, $\mathbf X=(1-\theta)\mathbf X+\theta\sum_{i=1}^K\lambda_ir_i\mathbf{X}\mathbf{\Pi}i^Tdiag({\rho_i^{-1}})$\\
18:\quad Update $D$\\
19:\quad $E_1=\sum_{i=1}^K\lambda_i<\mathbf{D},\mathbf{\Pi}_i>$ \\
20:\quad $dE=E_1-E_0$, $E_0=E_1$\\
21:\quad Compute the distance matrix $\mathbf{D}$ \\
22:\quad Update $\epsilon$ \\
23: \textbf{Until} $dE$ is convergent \\

24: Subroutine Iterative bregman projections()\\
25: Input: the distance matrix $D$, the parameter $\epsilon$,
    the density function $\mathbf{p}=\rho_k$,
    $\mathbf{q}=\mathbf{\varrho_{k}}$\\
26: Initiate $i=0$, $\mathbf{v}^i=\mathbf{1}$(all-1 vector)\\
27: Initiate $\mathbf\xi=e^{-\mathbf{D}/\epsilon}$\\
28: \textbf{Repeat}\\
29:\quad $\mathbf{u}^i=\frac{\mathbf{p}}{\mathbf{\xi}_k\mathbf{v}^i}$\\
30:\quad $i=i+1$,$\mathbf{v}^i=\frac{\mathbf{q}}{\mathbf{\xi}_k^\mathbf{T}\mathbf{u}^i}$\\
31:\textbf{Until} $\mathbf{\mu}^i$ is convergent\\
32:$\mathbf{\Pi}_k=diag(\mathbf{u}^i)\mathbf{\xi}_k diag({\mathbf{v}^i})$\\
   \rule{0.47\textwidth}{0.5pt}
  \caption{Pseudocode of multi-class blue noise algorithm}\label{fig:pseudocode}
\end{figure}


\subsection{Numerical Optimization}
Eq.~\ref{eq:combined-wasserstein-barycenter} is a unified variational model for single class sampling and multi-class sampling.
We only consider numerical optimization of Eq.~\ref{eq:combined-wasserstein-barycenter}
for multi-class sampling.
For every discrete probability measure $\mu_i\in\bar\mu$,
we represent it as:
\begin{equation}
 \mu_i=\sum\limits_{j=1}^{n_i}\rho_i^j\delta_{x_i^j} \quad s.t. \sum\limits_{j=1}^{n_i}\rho_i^j=1.
\end{equation}
where $\rho_i^j$ is probability at sampling point $x_i^j$ for the $i$th-class sampling
and $n_i$ is the number of sampling points for the $i$th-class sampling.
For convenience of representation and computation,
we extend the discrete probability measure $\mu_i$ to all class sampling points.
We define all sampling points as
\begin{equation}
 X=(X_1,X_2,...,X_N)^T=(x_1^1,...,x_1^{n_1},...,x_N^1,...,x_N^{n_N})^T.
\end{equation}
\begin{equation}
\mathbf{g_i}=(\underbrace{0,0,...,0}_{\sum\limits_{j=1}^{i-1}n_j},\rho_i^1,\rho_i^2,...,\rho_i^{n_i},\underbrace{0,0,...,0}_{\sum\limits_{j=i+1}^{N}n_j})^T
\end{equation}
Thus,
we can represent the discrete probability measure $\mu_i$ as:
\begin{equation}
  \mu_i=\sum\limits_{j=1}^{n}g_i^j\delta_{x_i^j}
\end{equation}
where $n=\sum\limits_{i=1}^N{n_i}$.
For blue noise sampling,
$\rho_i^j$ is a constant and $\rho_i^j=1/n_i$.
Thus, we need only to compute variables $X$ in Eq. ~\ref{eq:combined-wasserstein-barycenter} .

\begin{figure}[ht]
\centering
\subfigure[]{
\label{one-class-a}
   \includegraphics[width=.4\textwidth]{figure/one-class-1024-1.png}
}
\subfigure[]{
\label{one-class-b}
   \includegraphics[width=.45\textwidth]{figure/remp.png}
}
\caption{
Comparisons of CCVT, CCVT-PD and our algorithm for a single class blue noise sampling.
(a) shows the comparison for the case of constant density.
(b) shows the comparison on for the case of a quadratic density function. }
\label{one-class-sampling}
\end{figure}



\begin{table}
\caption{\label{comparing-table}
Comparison of several frequencies and spatial statistics of sampling patterns.
}
 \begin{tabular}{c|c|c|c|c|c}

 % after \\: \hline or \cline{col1-col2} %\cline{col3-col4}
  \hline
   Methods & $v_{eff}$ & $\Omega$ & $\delta_{min}$ & $\delta_{avg}$ & $Q_6$ \\ \hline
  {[Balzer et al. 2009]}& 0.88& 2.395 & 0.765 & 0.886 & 0.488 \\
  {[De Goes et al. 2012]} & 0.855 & 2.014 & 0.739 & 0.872 & 0.417 \\
   Our algorithm & 0.865 & 2.136 & 0.76 & 0.877 & 0.42 \\ \hline
  \end{tabular}\\
  Notes:$v_{eff} $ and $\Omega$ are the effective Nyquist and oscillation measures.
The value of $\delta_{min}$ corresponds to the Poisson disk radius,
and $\delta_{avg}$ roughly measures how uniformly the points are distributed~\cite{schlomer:2011:farthest}.
$Q_6$ measures the similarity of a point distribution to a hexagonal arrangement~\cite{kansal:2000:nonequilibrium}
\end{table}


Every probability measure $\nu_i$ in Eq.~\ref{eq:combined-wasserstein-barycenter} may be a continuous probability measure or a discrete probability measure.
If $\nu_i$ is a continuous probability measure,
we must discretize  $\nu_i$ as a discrete probability measure
to compute the integrations required in Eq. ~\ref{eq:combined-wasserstein-barycenter} by quadrature,
\begin{equation}\label{eq:discrete-prbability-measure}
\nu_i=\sum\limits_{j=1}^{m_i}\varrho_i^j\delta_{y_i^j} \quad s.t. \sum\limits_{j=1}^{m_i}\varrho_i^j=1
\end{equation}
where $\varrho_i^j=\varrho_i(y_i^j)V_{y_i^j}$
and $V_{y_i^j}$ is the area with density $\varrho_i(y_i^j)$ at the point $y_i^j$ when the density function
$\varrho_i$ is approximated with a positive piecewise constant function,
$\varrho_i$ is the probability density function of the probability measure $\nu_i$,
and $m_i$ is the number of discrete points of the $i$th-class probability measure $\nu_i$.
In this paper,
we apply a regular discretization with the same area for all sampling points $y_i^j$.
We also apply the same sampling points for all probability measures $\nu_i$,
which is
\begin{equation*}
m_i=m_k=m, y_i^j=y_k^j=y^j \quad{i\neq k}.
\end{equation*}
Thus, we use $\mathbf{Y}=(y^1,y^2,...,y^m)^T$ to represent all sampling points for probability measures $\nu_i$.
Based on discrete representation of probability measures in Eq.~\ref{eq:discrete-prbability-measure},
%and regularized Wasserstein distance (Eq.~\ref{eq:regulate-wasserstein-distance}),
the variational problem (Eq.~\ref{eq:combined-wasserstein-barycenter}) can be represented as:
\begin{equation}\label{eq:discrete-multi-problem}
  \begin{aligned}
  \mathbf{X}=\arg\min\limits_{\mathbf{X}}\sum\limits_{i=1}^K\lambda_i<\mathbf{D}_i,\mathbf{\Pi}_i> \\
    s.t.\ \ \mathbf{\Pi}_i\mathbf{1}=\mathbf{\varrho}_i,\ \mathbf{\Pi}_i^T\mathbf{1}=\mathbf{g}_i
  \end{aligned}
\end{equation}
where
% $\mathbf{X}=\{X_1,...,X_N\}$ is the set of all samples,
 $\mathbf{D}_i,\mathbf{\Pi}_i\in R_+^{n\times m}$,
 $\mathbf{D}_i$ is the distance matrix and $D_i^{(j,k)}=d(x_i^j,y^k)^p$,
 $\mathbf{\Pi}_i$ is the $i$th-class transport cost matrix and $\Pi_i(x_i^j,y^k)$ represents the mount of mass transported from $x_i^j$ to $y^k$ for the $i$-th class,
 and $<\cdot,\cdot>$ is Frobenius product of two matrixes.

 We proceed with sample generation by computing an optimal solution of the variational problem (Eq.\ref{eq:discrete-multi-problem}).
 We apply a loop iteration algorithm to extremize the variational problem by repeatedly performing a minimization step over positions $\mathbf{X}$ followed by a projection step over transport plans $\mathbf{\Pi}_k$.

 For a fixed set of points $\mathbf X$,
 we compute  relaxed transport plans through a  regularized Wasserstein distance (Eq.\ref{eq:regulate-wasserstein-distance}).
 For every transport plan matrix $\mathbf{\Pi}_k$,
 Eq.\ref{eq:regulate-wasserstein-distance} can be discretized as
 \begin{equation}\label{eq:discrete-regulate-wasserstein-barycenter}
   W_p^p(\mathbf{D}_i,\mathbf{\Pi}_i)=arg\min\limits_{\mathbf{\Pi}_i}<\mathbf{D}_i,\mathbf{\Pi}_i>-\epsilon H(\mathbf{\Pi}_i)
 \end{equation}
The optimal transport plan $\mathbf{\Pi}_i$ can be obtained by iterative Bregman projection~\cite{cuturi:2013:sinkhorn,benamou:2015:iterative}, and
\begin{equation}\label{eq:transport-plan}
  \mathbf{\Pi}_i=diag(\mathbf{u})\mathbf{\xi}_idiag(\mathbf{v})
\end{equation}
where $\mathbf{\xi}_i=e^{-\frac{\mathbf{D}_i}{\epsilon}}$,
$\mathbf{u}\in \mathbb{R}_+^{n}$ and $\mathbf{v}\in\mathbb{R}_+^{m}$.
More details can be found in~\cite{cuturi:2013:sinkhorn,cuturi:2013:fast,benamou:2015:iterative}



Suppose $\Pi_i(i=1,...,K)$ is fixed,
variational formula Eq.~\ref{eq:discrete-multi-problem}
is the sum of a convex quadratic function of $\mathbf{X}$ when $p=2$.
If we only update $\mathbf{X}_i$ at every iteration,
$\mathbf{X}$ can be obtained by Newton iterative method,
\begin{equation}\label{eq:position-iterative}
  \mathbf{X}\leftarrow(1-\theta)\mathbf{X}+\theta\mathbf{Y}\mathbf{\Pi}_i^T diag({\rho_i}^{-1}) \quad (0\leq\theta\leq 1)
\end{equation}
To improve the randomness of samples,
we apply a random Newton iterative method to update all sampling points $\mathbf{X}$ simultaneously,
\begin{equation}
\begin{split}
\mathbf{X}\leftarrow(1-\theta)X+\theta\sum_{i=1}^K\lambda_ir_i\mathbf{Y}\Pi_i^Tdiag({\rho_i}^{-1})\\
 s.t. \sum\limits_{i=1}^{K}r_i=1, r_i\geq 0
 \end{split}
\end{equation}

where $r_i$ is a random number.
Figure~\ref{fig:pseudocode}
shows the pseudocode of this schedule.

%To avoid the algorithm to stuck in local minima,
%a simulated annealing strategy is introduced to the gradient descent method.

\section{Results}

\begin{figure*}[htb]
  \centering
 \subfigure[{CCVT [Balzer et al. 2009]}]{
  \label{fig:eight:a}
  \includegraphics[width=0.48\textwidth]{figure/sampling-ratio-CCVT.png}}
   \subfigure[{Ours}]{
  \label{fig:eight:b}
  \includegraphics[width=0.48\textwidth]{figure/sampling-ratio-our.png}}

  \caption{Evaluation on sampling ratio.
  The number of sampling points is $1024$.
  From left to right, sampling ratios are $1/(4,8,16,32,64)$,
  and the number of sampling points is $1024$.
  When the sampling ratio is no more than $1/4$,
  blue noise property is preserved well in our approach.
    % ������������ӣ�ccvt������������Խ��Խ�ã�
  % �������ǵĶ��ԣ������ñ���С�ڵ���1/4ʱ�򣬲����Ⱦ�Ӱ�첻��
  }
  \label{sampling-ratio}
\end{figure*}
\begin{figure*}[htb]
\centering
\subfigure[{ Dart throwing [Wei 2010]}]{
   \includegraphics[width=0.38\textwidth]{figure/2-class-weiliyi.png}
   \label{two-class-sampling-a}
}%
\hspace{0.1in}
\subfigure[{ Ours: $\lambda_{1,2,3}=(1,1,1)/3$}]{
   \includegraphics[width=.38\textwidth]{figure/2-111.png}
   \label{two-class-sampling-a}
}\\
\subfigure[{ Ours: $\lambda_{1,2,3}=(1,1,2)/4$}]{
   \includegraphics[width=.38\textwidth]{figure/2-112.png}
   \label{two-class-sampling-c}
}%
\hspace{0.1in}
\subfigure[{ Ours: $\lambda_{1,2,3}=(1,1,6)/8$}]{
   \includegraphics[width=.38\textwidth]{figure/2-116.png}
}%
\caption{The comparison of Wei's algorithm and our algorithm for two-class blue noise sampling.
In our algorithm, $\lambda_1$,$\lambda_2$,$\lambda_3$ are the weighted parameters for red samples, blue samples and the combined samples.
The number of samples for each class is 1000.
The weighted parameter $\lambda_i$ makes an influence on the trade-off between good distribution of the samples belonging to each individual class and good distribution of total samples.
Large $\lambda_i$ preserves the blue noise property of the $i$-$th$ well.
}\label{two-class-sampling}
\end{figure*}

We ran our algorithm on a variety of inputs:
from constant density to stripe and point set surface.
Various illustrations based on regularity and spectral analysis are
used throughout the paper to allow easy evaluation of our algorithm and  demonstrate how they  are compared to previous works.
We use methods in~\cite{schlomer:2011:accurate} and~\cite{wei:2011:differential} to analyze the
spectrum properties of sample distributions in two dimensional space and two dimensional manifold separately.
For spatial uniformity,
the relative radius~\cite{Lagae:2008:CPDD} is used to analyze the spatial uniformity.



\textbf{Single class sampling.}
Blue noise point distribution for a constant density
shows a characteristic blue noise profile in spectral space
and a typical spatial arrangement.
In Fig.~\ref{one-class-sampling},
we compare our algorithm with CCVT~\cite{balzer:2009:capacity} and CCVT-PD~\cite{de:2012:blue} method for the case of constant density in spectral space.
We also provide evaluations of the spatial properties in Table~\ref{comparing-table}.
Furthermore,
we also make an evaluation of blue noise sampling for an intensity ramp in Fig.~\ref{one-class-sampling}
by counting the number of points for each
quarter of the ramp.
It can be shown that our algorithm is comparable to CCVT~\cite{balzer:2009:capacity} and CCVT-PD~\cite{de:2012:blue} for both the cases of constant density and the intensity ramp.

Since the involved numerical optimization is based on discrete representation of sampled density functions in our approach,
we evaluate the effect of sampling ratio ($n/m$) on blue noise property by comparing our approach with CCVT~\cite{balzer:2009:capacity} in Fig.~\ref{sampling-ratio}.
When the sampling ratio is equal to $1/4$,
sampling on CCVT~\cite{balzer:2009:capacity} doesn't show blue noise property well.
With the decreasing sampling ratio, blue noise property is shown more and more well.
However,
the sampling ratio doesn't make an important effect on our approach.% when the sampling ratio is no more than $1/4$.
The reason is that the relaxed transport plan makes more original points exert influence on one sampling point in our approach, other than only nearest original points play the role in CCVT~\cite{balzer:2009:capacity}.



\textbf{Multi-class sampling on constant density.}
To evaluate blue noise distribution for multi-class sampling,
we first analyse the simplest case where all classes have the constant density function and same number of samples.
Fig.~\ref{two-class-sampling}, Fig.~\ref{three-class-sampling},
and Fig.~\ref{seven-class-sampling} illustrate the result generated by our method, Wei's ~\shortcite{wei:2010:multi}, and SPH~\cite{jiang:2015:blue} on two-class, three-class, and seven-class sampling separately.
In two-class sampling (Fig.~\ref{two-class-sampling}),
we illustrate the influence of the weighted parameter $\lambda_i$ on the trade-off between good distribution of the samples belonging to each individual class and good distribution of total samples.
Large $\lambda_i$ preserves the blue noise property of the $i$-$th$ well.
Depending on which distribution is more important,
$\lambda_i$ can be chosen for different applications.
When $\lambda_1+\lambda_2=\lambda_3$,
the balance between good distribution of the samples belonging to each individual class and good distribution of total samples
is achieved in Fig.~\ref{two-class-sampling-c}.
Fig.~\ref{two-class-sampling-a} and Fig.~\ref{two-class-sampling-c},
show that our results are comparable to that of Wei's.
In three-class sampling (Fig.~\ref{three-class-sampling}),
we evaluate blue noise distribution of samples of different combined classes.
Compared with Wei's method~\cite{wei:2010:multi} and SPH~\cite{jiang:2015:blue},
 blue noise profile of the combined class with two individual class is improved in our method
while blue noise distribution of each individual class is preserved.
In seven-class sampling (Fig.~\ref{seven-class-sampling}),
blue noise profile  is still kept for samples of each individual class and total class.
It shows that our method is feasible for more numbers of class sampling.

% ��һ��
To evaluate efficiency of relaxing conflicts of desired centroid locations in our approach,
we compare our approach with Wei's method~\cite{wei:2010:multi} on the
difference between centroid locations of voronoi diagrams for single class samples and multi-class samples in Fig.~\ref{conflict-evaluation}.
We apply conflicts ratio to evaluate the difference between desired centroid locations.
Conflicts ratio is defined as
\begin{equation*}
  R=\frac{\sum\limits_{i=1}^N\sum\limits_{j=1}^{n_i}||C_{x_i^j}^I-C_{x_i^j}^C||}{\sum\limits_{i=1}^N\sum\limits_{j=1}^{n_i}D(x_i^j)}
\end{equation*}
where $C_{x_i^j}^I$ and $C_{x_i^j}^C$ are centroid locations of voronoi diagrams of the sample ${x_i^j}$ for individual class samples and combined class samples,
,and $D(x_i^j)$ is the distance between the sample $x_i^j$ and its nearest neighbor sample,
$n$ is the number of samples.
Compared with Wei's approach~\shortcite{wei:2010:multi},
conflicts ratio in our approach is decreased approximately by $50\%$.
It shows that our approach avoids the failure case of
multi-class sampling on traditional Lloyd relaxation [Wei 2010],
and is an efficient relaxation method for multi-class sampling.

\begin{figure*}[htb]
  \centering
  % Requires \usepackage{graphicx}\textwidth
  \subfigure[{Dart throwing [Wei 2010]}]{

  \includegraphics[width=0.9\textwidth]{figure/3-weiliyi.png}}

  \subfigure[{SPH [Jiang et al. 2015]}]{
  \includegraphics[width=0.9\textwidth]{figure/3-sph.png}}
    \subfigure[Ours]{

  \includegraphics[width=0.9\textwidth]{figure/3-our.png}}
  \caption{The comparison of Wei's algorithm, SPH and our algorithm for three-class blue noise sampling.
  $\lambda_{1,2,3,4,5,6,7}=(1,1,1,2,2,2,9)/18$.
  $\lambda_1$,$\lambda_2$ and $\lambda_3$ are weighted parameters for each individual class.
  $\lambda_1$,$\lambda_2$ and $\lambda_3$ are weighted parameters for combined classes with two individual class.
  $\lambda_8$ is the weighted parameters for the total samples.
  The number of samples of each individual class is $1024$.}
  \label{three-class-sampling}
\end{figure*}

\begin{figure*}[]
  \centering
  % Requires \usepackage{graphicx}\textwidth
 % \subfigure[{Dart throwing [Wei 2010]}]{
 % \label{fig:eight:a}
 % \includegraphics[width=0.8\textwidth]{figure/7-class-weiliyi-1024.png}}
%  \subfigure[Our method]{
 %  \label{fig:eight:b}
  \includegraphics[width=1.0\textwidth]{figure/7-class-our-1024.png}

  \caption{Seven-class blue noise sampling on our algorithm.
  $\lambda_{1,2,3,4,5,6,7,8}=(1,1,1,1,1,1,1,7)/14$.
  $\lambda_1,\cdots,\lambda_7$ is weighted parameters for each individual class,
  $\lambda_8$ is the weighted parameter for the total set.
  The number of samples of each individual class is $1024$.}\label{seven-class-sampling}
\end{figure*}

\begin{figure}[htb]
    \centering
    \includegraphics[width=0.4\textwidth]{figure/conflict-ratio.png}
    \caption{The evaluation for conflicts of desired voronoi centroid locations of individual classes and combined class.
    }
    \label{conflict-evaluation}
\end{figure}

\begin{figure}[htb]
  \centering
 \subfigure[Density function]{
  \label{fig:eight:a}
  \includegraphics[width=0.4\textwidth]{figure/adapt-color.png}}
   \subfigure[{Dart throwing [Wei 2010]}]{
  \label{fig:eight:b}
  \includegraphics[width=0.4\textwidth]{figure/adapt-liyiwei.png}}
   \subfigure[Ours]{
  \label{fig:eight:c}
  \includegraphics[width=0.4\textwidth]{figure/adapt-our.png}}
  \caption{Adaptive two-class sampling of a non-uniform density function with 1000 points for each class.
  The percentages in each quarter indicate ink density of different color in the image. In contrast,
  our results show precise adaptation for every single sampling and total sampling.}\label{adaptive sampling}
\end{figure}

\begin{figure}[htb]
  \centering
  \includegraphics[width=0.45\textwidth]{figure/bunny-sampling.png}
  \caption{Two-class blue noise sampling on a point set surface.
  $\lambda_{1,2,3}=(1,1,2)/4$, and the number of samples belonging to each class is $2000$.  }\label{bunny-sampling}
\end{figure}


\textbf{Adaptive multi-class sampling on ramp.}
To evaluate multi-class blue noise sampling on non-uniform density function,
we applied two intensity ramp and count the number of points for each quarter of the ramp.
Fig.~\ref{adaptive sampling} shows samples generated by
our method and Wei's~\cite{wei:2010:multi}.
While samples of every individual class and the combined class represent approximately the right counting of points per quarter,
it can be seen that our results show noticeably less noise.

\textbf{Multi-class sampling on point set surface.}
Our method can be directly applied to multi-class sampling on point set surface.
We assume that original points represent a discrete probability measure $\nu=\sum\limits_{k=1}^{m}\varrho_k\delta_{y^k}$ as Eq.~\ref{eq:discrete-prbability-measure} on the surface.
$\varrho_k$ is set as the normalized area element at the point $y^k$ for multi-class blue noise sampling with constant density function.
$D_i^{(j,k)}$ is the geodesic distance.
For convenience,
we applied the Euclidean distance instead of the geodesic distance in this paper.
Samples are initialized on the surface.
After each iteration,
every sample is mapped back onto the surface by
moving least square projection~\cite{alexa:2001:point}.
In Fig.~\ref{bunny-sampling},
we make a two-class blue noise sampling on a bunny model,
and show the Differential Domain Function~\cite{wei:2011:differential} to demonstrate blue noise profile of different class sample distributions.

\textbf{Multi-class blue noise sampling with different sizes.}
Analogous to setting different radius of Possion disks for different class in dart throwing,
different sizes of samples can be set  via discrete probability measure $\mu_i$ for different classes in our method.
Suppose that the same constant density function $\varrho_i (1\leq i\leq N)$ is used for different classes,
we set the discrete probability density as $\rho_i^j=1/n_i$ for every individual class and $\rho_i^j=1/Nn_i$ for every combined class,
where $N$ is the number of individual classes of the combined class.
When two-class sampling is done with $N_1$ and $N_2$ samples ($N_1>N_2$) ,
every sample belonging to the $2$-th class is given more measure than every sample belonging to the $1$-th in the combined class.
When the density function of the combined class is a constant function,
the $2$-th sample is with larger size than the $1$-th sample.
The radius of every sample can be approximated as:
\begin{equation*}
  r_i=\lambda r_{i_{max}}/n
\end{equation*}
where $r_{i_max}$ is the maximun possible disk radius of the $i$-th class,
$\lambda$ is a relative radius parameter.
In our experiments,
we found that there is on overlap between any pair samples when $0\leq\lambda\leq 0.5$.
In Fig.~\ref{adaptive sampling},
our method is applied for three-class blue noise sampling with different size on a square and a two dimensional manifold.



\textbf{Complexity.}
One iteration of our algorithm involves iterative bregman projection.
The time complexity depends on what classes are involved in the framework.
For a single class sampling,
the time complexity is $O(MN)$.
Although this is worse than $O(NlogN+M)$,
iterative bregman projection can be easily parallelized on the GPU.
When only a total set is used as a combined class,
the time complexity is $O(2MN)$.
If all $2^n-1$ classes are involved,
the complexity is $O(2^{n-1}MN)$.
For a general application,
only a combined class of the total set is enough to generate a good point distribution.
Thus, it is significantly better than CapCVT~\cite{chen:2012:variational} which is $O(nMN+nN^2)$.
While our algorithm performs well for small datasets,
our method is limited by its memory requirements $O(2MN)$ for large datasets, such as the experiment in Fig.~\ref{Color stippling}.

\textbf{Performance.}
% �õ�ʲô�豸
% ��Ե�������ʲô��������ʲô
% ��Զ�������ʲô��������ʲô
All of our performance are locked on a workstation with Intel Xeon 3.50GHz
dual-core CPUs and 32GB memory, and NVIDIA Quadro K5000 GPU with 2GB memory.
For single blue noise sampling,
we compare our method with traditional CCVT~\cite{balzer:2009:capacity}
and power diagrams method~\cite{de:2012:blue} on running time in Table~\ref{time-table-singleclass}
Since the most time is spend for integral computation in power diagrams method~\cite{de:2012:blue} and distance matrix computation in our approach,
CCVT~\cite{balzer:2009:capacity} performs more better in running time than these two method when sampling rate is lower than $1/64$.
With the increasing of sampling points or the decreasing of sampling rate,
our approach outperforms CCVT~\cite{balzer:2009:capacity} and power diagrams~\cite{de:2012:blue}.
With the increasing of sampling points,
our approach is more and more faster than CCVT~\cite{balzer:2009:capacity},
and is two times faster than CCVT~\cite{balzer:2009:capacity} at least.
With the increasing of sampling points,
our approach is faster than power diagrams method~\cite{de:2012:blue} by over $10\%$.
For mutli-class blue noise sampling,
we compare our method with Wei's method~\cite{wei:2010:multi} on running time in Table~\ref{time-table-multiclass}.
Our approach is two times faster than Wei's method~\cite{wei:2010:multi} at least.
So far,
we use a global distance matrix in our algorithm.
More memory space is needed to store the matrix in our algorithm than other algorithm.
In every iteration,
updating the matrix spends most of running time.
Thus, our approach faces a challenge when numbers of original points and sampling points are large.
However, we also notice that the matrix is a sparse matrix in general.
Thus, we can furthermore improve the performance of our approach by making use of a sparse distance matrix.

\begin{table}
\center
\caption{\label{time-table-singleclass}
Comparison of sampling times of multi-class blue noise sampling.
Original points are generated from a constant density function.
The resolution of original points is $128\times 128$.
}

 \begin{tabular}{c|c|c|c|c}
 {Sampling} & {Sampling} &
 %\multirow{2}*{Original Points} &
 \multicolumn{3}{c}{ Times(s)} \\ \cline{3-5}
 \multirow{2}*{Points} & \multirow{2}*{Rate}
   & [Balzer et & [De Goes &  \multirow{2}* {Ours} \\
   &  & al. 2009] &  et al. 2012] & \\ \hline
 % after \\: \hline or \cline{col1-col2} %\cline{col3-col4}

   32 & 1/512&   0.203 & $0.325$ & 0.404 \\ \hline
   64 & 1/256& 0.187 & 0.407 & 0.346  \\ \hline
   128 & 1/128&  0.202 & 0.436 & 0.433  \\ \hline
   256 & 1/64& 0.375 & 0.568 & 0.474  \\ \hline
   512 & 1/32& 1.076 & 0.739 & 0.669  \\ \hline
   1024 & 1/16& 2.184 & 1.137 & 1.048  \\ \hline
   2048 & 1/8&  8.408 & 2.269 & 2.025  \\ \hline
   4096 & 1/4 & 30.841 & 4.406 & 4.005 \\ \hline

  \end{tabular}\\
\end{table}


\begin{table}
\center
\caption{\label{time-table-multiclass}
Comparison of sampling times of single blue noise sampling.
The number of sampling points of every individual class is 512.
The sampling rate is $1/16$ for every individual class sampling.
}

 \begin{tabular}{c|c|c}
 {Number of} &
 %\multirow{2}*{Original Points} &
 \multicolumn{2}{c}{ Times(s)} \\ \cline{2-3}
  Class & [Wei 2010] & Ours \\ \hline


 % after \\: \hline or \cline{col1-col2} %\cline{col3-col4}

   2 & 34.769 & 14.670  \\ \hline
   3 & 151.727 & 37.553   \\ \hline
   7 & 414.031 & 212.133 \\ \hline


  \end{tabular}\\
\end{table}
%We measure the timing via a uniform sampling.
%The number of all samples is 2048.
%The number of discrete points $M=128\times 128$.
%Only total set is considered as a combined class.
%Table. shows the computation times of our algorithm.

\section{Application}

\begin{figure*}[htb]
  \centering

  \includegraphics[width=0.9\textwidth]{figure/noisy.png}
  \caption{Surface reconstruction of noisy point clouds.
  (a) is original point clouds with $8856$ points.
  (c) is mesh reconstruction result on $3946$ sampling points.
  (b) is rendering result with point clouds (a) and mesh model (c).
  (d) is the close-up view of (c) at the square mark.
   }\label{surface-reconstruction}
\end{figure*}

\begin{figure*}[htb]
  \centering
   \begin{minipage}{1\textwidth}
  \includegraphics[width=0.245\textwidth]{figure/gs-123.png}
  %\end{minipage}
  %  \begin{minipage}{0.25\textwidth}
  \includegraphics[width=0.245\textwidth]{figure/gs-1.png}
  %\end{minipage}
   % \begin{minipage}{0.23\textwidth}
  \includegraphics[width=0.245\textwidth]{figure/gs-2.png}
  %\end{minipage}
  %  \begin{minipage}{0.23\textwidth}
  \includegraphics[width=0.245\textwidth]{figure/gs-3.png}
  \end{minipage}
  \caption{Object placement on a complex geometry model.
  The number of samples of each class is $200$, $1000$ and $1000$.}\label{gs-sampling}
\end{figure*}


\textbf{Surface reconstruction of noisy point clouds.}
Surface reconstruction of noisy point clouds is a challenging problem.
In this paper,
we take original point clouds with noise as a discrete probability measure of a two dimensional surface,
and take samples on the surface as another discrete probability measure.
Samples on the surface is the Wasserstein barycenter of original point clouds.
We can apply a single class blue noise sampling to generate samples on the surface.
Then triangulations are implemented by the open source code provided by Rahmani et al.~\cite{rahmani:2014:hopc}.
Fig.~\ref{surface-reconstruction} shows an example for surface reconstruction of point clouds with heavy noise.
Since the entropic regular term in the Wasserstein barycenter plays an role for smoothing of the transport plan with hard constrain (Eq.~\ref{eq:transform-plan}),
our approach can generate an smooth surface.

\textbf{Object distribution.}
Object placement without regularity artifacts plays an important role in texture synthesis.
Blue noise sampling provides a feasible approach for such distribution.
However,
how to place multi-class objects with different size on the surface is a considerable problem .
Our approach can be directly applied on this problem.
Figure~\ref{adaptive-size-sampling} shows an example for placing three class objects with different sizes on a square.
Figure~\ref{adaptive-size-sampling} and Figure~\ref{gs-sampling} show two examples for placing three class objects with different sizes on two dimensional surfaces.

\textbf{Color stippling.}
Color stippling is another very important application via blue noise sampling.
We can apply our method to multi-class color stippling on CMY-RGB-K color model using the input image as a weighted density filed,
The weighted value represents the importance of each color channel and is used to set the number of each color dots.
More dots are placed in the
area where the corresponding color is dominant.
The discrete probability value $\bar\rho_k$ is set as a constant $1/N_k$ for every class.
As shown in Figure~\ref{Color stippling},
our method can achieve reasonable color stippling.
Note that the visual quality depends on color decomposition and blending method.

\textbf{Visual abstraction}
Scatterplots are widely used to visualize scatter dataset.
When multi-class scattered points are shown within a single scatterplot view, heavy overdraw of different class points makes it inefficient for analysis (Figure~\ref{visual-abstraction}).
We apply our algorithm to reduce the overdraw of original scattered points and preserves the point distributions in Figure~\ref{visual-abstraction}.
Compared with dart throwing method~\cite{wei:2010:multi,chen:2014:visual},
it is not necessary to compute a continuous density function for each class in our method.



\begin{figure}[htb]
  \centering
    \includegraphics[width=0.48\textwidth]{figure/corner-2.png}
   \caption{Color stippling by using the RGB-CMYK color model.
  The number of samples is 48k.  }\label{Color stippling}
\end{figure}

\begin{figure}[htb]
  \centering
  % Requires \usepackage{graphicx}\textwidth
 % \subfigure[]{
 % \label{fig:eight:a}
 % \includegraphics[width=0.10\textwidth]{figure/magic-cbue-original.png}}
 % \subfigure[]{
 %  \label{fig:eight:b}
 % \includegraphics[width=0.25\textwidth]{figure/magic-cube.png}}\\
    \subfigure[]{
  \label{fig:eight:a}
  \includegraphics[width=0.23\textwidth]{figure/scatter-plot-1.png}}
  \subfigure[]{
   \label{fig:eight:b}
  \includegraphics[width=0.23\textwidth]{figure/scatter-plot-sampling-1.png}}

  \caption{Visual abstraction of multi-class scatterplots.
   The numbers of original points in(a) are 14880, 14130, and 12600 for yellow, red and blue points.
   The numbers of samples in (b) are 496,471,420.}
   \label{visual-abstraction}
\end{figure}

\section{Conclusion and future work}
We have presented a new multi-class blue noise sampling method by
throwing samples as a  Wasserstein barycenter of multi-class density function.
Our approach introduces weighted parameters to explicitly adjust point distributions of each individual class and the combined classes.
By adjusting the discrete density value at samples,
we are able to generate multi-class samples with different sizes.
%Our approach can also be directly applied to multi-class sampling on point set surface.
Our approach is also easy to implement in parallel, which ensures the efficiency of computations.
We have also performed applications on object placement and color stippling.

In the future,
an interesting problem would be to study how to extend our approach to generate point distributions with multi-given target characteristics.
Another possible direction is to study how to improve the efficiency for large databases.


\section*{Acknowledgements}

\bibliographystyle{acmsiggraph}
\nocite{*}
\bibliography{template}
\end{document}
