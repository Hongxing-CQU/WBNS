\section{Application}

\begin{figure*}[htb]
  \centering

  \includegraphics[width=0.9\textwidth]{figure/noisy.png}
  \caption{Surface reconstruction of noisy point clouds.
  (a) is original point clouds with $8856$ points.
  (c) is mesh reconstruction result on $3946$ sampling points.
  (b) is rendering result with point clouds (a) and mesh model (c).
  (d) is the close-up view of (c) at the square mark.
   }\label{surface-reconstruction}
\end{figure*}

\begin{figure*}[htb]
  \centering
   \begin{minipage}{1\textwidth}
  \includegraphics[width=0.245\textwidth]{figure/gs-123.png}
  %\end{minipage}
  %  \begin{minipage}{0.25\textwidth}
  \includegraphics[width=0.245\textwidth]{figure/gs-1.png}
  %\end{minipage}
   % \begin{minipage}{0.23\textwidth}
  \includegraphics[width=0.245\textwidth]{figure/gs-2.png}
  %\end{minipage}
  %  \begin{minipage}{0.23\textwidth}
  \includegraphics[width=0.245\textwidth]{figure/gs-3.png}
  \end{minipage}
  \caption{Object placement on a complex geometry model.
  The number of samples of each class is $200$, $1000$ and $1000$.}\label{gs-sampling}
\end{figure*}


\textbf{Surface reconstruction of noisy point clouds.}
Surface reconstruction of noisy point clouds is a challenging problem.
In this paper,
we take original point clouds with noise as a discrete probability measure of a two dimensional surface,
and take samples on the surface as another discrete probability measure.
Samples on the surface is the Wasserstein barycenter of original point clouds.
We can apply a single class blue noise sampling to generate samples on the surface.
Then triangulations are implemented by the open source code provided by Rahmani et al.~\cite{rahmani:2014:hopc}.
Fig.~\ref{surface-reconstruction} shows an example for surface reconstruction of point clouds with heavy noise.
Since the entropic regular term in the Wasserstein barycenter plays an role for smoothing of the transport plan with hard constrain (Eq.~\ref{eq:transform-plan}),
our approach can generate an smooth surface.

\textbf{Object distribution.}
Object placement without regularity artifacts plays an important role in texture synthesis.
Blue noise sampling provides a feasible approach for such distribution.
However,
how to place multi-class objects with different size on the surface is a considerable problem .
Our approach can be directly applied on this problem.
Figure~\ref{adaptive-size-sampling} shows an example for placing three class objects with different sizes on a square.
Figure~\ref{adaptive-size-sampling} and Figure~\ref{gs-sampling} show two examples for placing three class objects with different sizes on two dimensional surfaces.

\textbf{Color stippling.}
Color stippling is another very important application via blue noise sampling.
We can apply our method to multi-class color stippling on CMY-RGB-K color model using the input image as a weighted density filed,
The weighted value represents the importance of each color channel and is used to set the number of each color dots.
More dots are placed in the
area where the corresponding color is dominant.
The discrete probability value $\bar\rho_k$ is set as a constant $1/N_k$ for every class.
As shown in Figure~\ref{Color stippling},
our method can achieve reasonable color stippling.
Note that the visual quality depends on color decomposition and blending method.

\textbf{Visual abstraction}
Scatterplots are widely used to visualize scatter dataset.
When multi-class scattered points are shown within a single scatterplot view, heavy overdraw of different class points makes it inefficient for analysis (Figure~\ref{visual-abstraction}).
We apply our algorithm to reduce the overdraw of original scattered points and preserves the point distributions in Figure~\ref{visual-abstraction}.
Compared with dart throwing method~\cite{wei:2010:multi,chen:2014:visual},
it is not necessary to compute a continuous density function for each class in our method.



\begin{figure}[htb]
  \centering
    \includegraphics[width=0.50\textwidth]{figure/corner-2.png}
   \caption{Color stippling by using the RGB-CMYK color model.
  The number of samples is 48k.  }\label{Color stippling}
\end{figure}

\begin{figure}[htb]
  \centering
  % Requires \usepackage{graphicx}\textwidth
 % \subfigure[]{
 % \label{fig:eight:a}
 % \includegraphics[width=0.10\textwidth]{figure/magic-cbue-original.png}}
 % \subfigure[]{
 %  \label{fig:eight:b}
 % \includegraphics[width=0.25\textwidth]{figure/magic-cube.png}}\\
    \subfigure[]{
  \label{fig:eight:a}
  \includegraphics[width=0.4\textwidth]{figure/scatter-plot-1.png}}
  \subfigure[]{
   \label{fig:eight:b}
  \includegraphics[width=0.4\textwidth]{figure/scatter-plot-sampling-1.png}}

  \caption{Visual abstraction of multi-class scatterplots.
   The numbers of original points in(a) are 14880, 14130, and 12600 for yellow, red and blue points.
   The numbers of samples in (b) are 496,471,420.}
   \label{visual-abstraction}
\end{figure}


